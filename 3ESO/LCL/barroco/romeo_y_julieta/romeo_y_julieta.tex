\documentclass[12pt,a4paper]{article}
\usepackage[spanish]{babel}
\usepackage[utf8]{inputenc}

\usepackage[right=2cm,left=3cm,top=2cm,bottom=2cm,headsep=0cm,footskip=0.5cm]{geometry}
\usepackage{graphicx,subfigure}
\usepackage{amsmath}
\usepackage{underscore}
\usepackage{array}
\usepackage{yhmath}
\usepackage{adjustbox}
\usepackage{multirow}

\title{\textbf{Análisis de \textit{Romeo y Julieta}}}
\author{Alberto Navalón Lillo}

\begin{document}

\maketitle
\tableofcontents

\section{Introducción}

\textit{Romeo y Julieta} (del original, en inglés: \textit{The Most Excellent and Lamentable Tragedy of Romeo and Juliet}), es una tragedia del dramaturgo inglés William Shakespeare, escrita en 1597. Esta obra, no obstante, no entra dentro del periodo denominado \textbf{barroco inglés}, que se dice comprender desde el Gran Incendio de Londres, en 1666, hasta el firmado del Tratado de Utrecht, en 1715. Sin embargo, se podría considerar una muestra de un barroco temprano si aplicamos los estándares españoles, ya que Shakespeare fue contemporáneo a otros autores (dramaturgos y poetas) españoles de gran renombre, como Miguel de Cervantes, Lope de Vega, Francisco de Quevedo o Luis de Góngora.\\

Puede parecer que esta obra no tiene nada que ver, ni puede compartir características con la literatura barroca española, dado que España no influía mucho artísticamente en Inglaterra. Pero si recordamos que \textit{Romeo y Julieta} está basado en un cuento italiano escrito por Mateo Bandello, cuya traducción al inglés, realizada por Arthur Brooke, se basó en la traducción francesa hecha por Pierre Boaistuau. En este enredo de traducciones aparecen Italia y Francia, países donde España tenía una influencia, artísticamente hablando, bastante más significativa, de forma que es posible que alguna de las cualidades de la literatura barroca española esté presente en alguna de esas conexiones, y haya influenciado a las demás. De todas formas, tampoco es estrictamente necesario ver presentes las características de nuestra literatura, dado que la suya, especialmente la shakespeariana, tenía otras particulares, también muy dignas de ser estudiadas.\\

Esta es la historia de dos jóvenes, de familias rivales, que se enamoran, y deciden casarse y vivir juntos clandestinamente. Sin embargo, una serie de acontecimientos y decisiones, tomadas no solo por los dos amantes, sino también por el resto de personajes, llevan a la pareja a la muerte, y a sus familiares, a algunos a la muerte, y a otros al arrepentimiento. En este argumento, podemos encontrar, a simple vista, temas que ya hemos tratado con las otras obras, como es el caso del amor y del honor, temas que se tratarán con más profundidad después. Por tanto, ya hemos encontrado una similitud entre la obra dramática lopesca, de carácter barroco español, y la shakespeariana, que se podría considerar de carácter prebarroco inglés, aunque quizá sea más correcto incluirla en el periodo del teatro isabelino, que se reserva casi en su totalidad a la obra dramática de William Shakespeare, respecto a importancia.\\

\section{Características de la obra}

Dentro de \textit{Romeo y Julieta}, por ejemplo, no encontramos una figura como la del bufón, típica de otras obras shakespearianas, como \textit{Hamlet}, que da un punto cómico a una acción casi en su totalidad trágica. Pero, a pesar de no tener una figura como la del bufón, encontramos pasajes cómicos representados por los propios personajes que influyen en el desarrollo de la obra, sin recurrir a un personaje terciario. Este es un ejemplo que me llamó mucho la atención:\\

\noindent\textsc{Mercucio}\\
\indent\textit{¡Qué ingenio! Sígueme la broma hasta gastar el zapato, que, cuando suelen gastarse las suelas, te quedas desolado por el pie.}\\
\textsc{Romeo}\\
\indent\textit{¡Ah, broma descalza, que ya no con-suela!}\\
\textsc{Mercucio}\\
\indent\textit{Sepáranos, Benvolio: me flaquea el sentido.}\\
\textsc{Romeo}\\
\indent\textit{Mete espuelas, mete espuelas o te gano.}\\
\textsc{Mercucio}\\
\indent\textit{Si hacemos carrera de gansos, pierdo yo, que tú eres más ganso con un solo sentido que yo con mis cinco.}\\
\indent\textit{¿Estamos empatados en lo de «ganso»?}\\
\textsc{Romeo}\\
\indent\textit{Empatados, no. En lo de «ganso» estamos engansados.}\\
\textsc{Mercucio}\\
\indent\textit{Te voy a morder la oreja por esa.}\\
\textsc{Romeo}\\
\indent\textit{Ganso que grazna no muerde.}\\
\textsc{Mercucio}\\
\indent\textit{Tu ingenio es una manzana amarga, una salsa picante.}\\
\textsc{Romeo}\\
\indent\textit{¿Y no da sabor a un buen ganso?}\\
\textsc{Mercucio}\\
\indent\textit{¡Vaya ingenio de cabritilla, que de una pulgada se estira a una vara!}\\
\textsc{Romeo}\\
\indent\textit{Yo lo estiro para demostrar que a lo ancho y a lo largo eres un inmenso ganso.}\\
\textsc{Mercucio}\\
\indent\textit{¿A que más vale esto que gemir de amor? Ahora eres sociable, ahora eres Romeo, ahora eres quien eres, por arte y por naturaleza, pues ese amor babeante es como un tonto que va de un lado a otro con la lengua fuera para meter su bastón en un hoyo.}\\
\textsc{Benvolio}\\
\indent\textit{¡Para, para!}\\
\textsc{Mercucio}\\
\indent\textit{Tú quieres que pare mi asunto a contrapelo.}\\
\textsc{Benvolio}\\
\indent\textit{Si no, tu asunto se habría alargado.}\\
\textsc{Mercucio}\\
\indent\textit{Te equivocas: se habría acortado, porque ya había llegado al fondo del asunto  y no pensaba seguir con la cuestión.}\\
\textsc{Romeo}\\
\indent\textit{¡Vaya aparato!}
\begin{flushright}
	\textsc{Shakespeare}, William. \textit{Romeo y Julieta}. Ed. Austral (p. 89-91).\\
\end{flushright}

En este fragmento podemos observar este tipo de comicidad dentro de una acción tan seria, emocional y trágica como es \textit{Romeo y Julieta}, toque característico que aporta Shakespeare a este tipo de obras. Lope de Vega también hace algo parecido, en forma de «mezcla de lo trágico y lo cómico». Sin embargo, Lope lo hace con tal de construir una semejanza lo más realista posible con la vida real, mientras que Shakespeare lo hace con tal de evitar una acción trágica en su totalidad, y no aburrir a los espectadores.. Pero lo que más destaca aquí es el ingenioso juego de palabras (pies, gansos...) que se utiliza para llegar a este fin cómico. Por tanto, este pasaje tiene un importante efecto cómico, y a su vez una gran riqueza poética, propia de los modelos conceptistas, presentes en la obra poética de autores españoles de gran renombre, como Francisco de Quevedo y Luis de Góngora. Con esto, tendemos otro vínculo entre la literatura shakespeariana y la española, y más concretamente, la lírica.\\

También destacamos la versificación del lenguaje, y el alt nivel poético de las voces cultas, en este caso, de los dos amantes. Esta es una característica común tanto del teatro lopesco como del teatro shakespeariano e isabelino. Sin embargo, una de las características particulares a Shakespeare, que parte de esta idea de versificar el lenguaje, es el mezclar el verso con la prosa. Asimismo, podemos encontrar pasajes en la obra tanto en prosa:\\

\textsc{Mercucio}\\
\indent\textit{Gran rey de los gatos, tan sólo perderle el respeto a una de tus siete vidas y, según como me trates desde ahora, zurrar a las otras seis. ¿Quieres sacar ya de cuajo tu espada? Deprisa, o la mía te hará echar el cuajo.}\\
\begin{flushright}
	\textsc{Shakespeare}, William. \textit{Romeo y Julieta}. Ed. Austral (p. 103).\\
\end{flushright}

Como en verso:\\

\textsc{Julieta}\\
\indent\textit{¡Condenada vieja! ¡Perverso demonio!}\\
\indent\textit{¿Qué es más pecado? ¿Tentarme al perjurio}\\
\indent\textit{o maldecir a mi esposo con la lengua}\\
\indent\textit{que tantas veces lo ensalzó}\\
\indent\textit{con desmesura? Vete, consejera.}\\
\indent\textit{Tú y mis pensamientos viviréis como extraños.}\\
\indent\textit{Veré qué remedio puede darme el fraile;}\\
\indent\textit{si todo fracasa, habré de matarme.}\\
\begin{flushright}
	\textsc{Shakespeare}, William. \textit{Romeo y Julieta}. Ed. Austral (p. 131).\\
\end{flushright}

\subsection{Los temas de la obra}

Como se ha mencionado en la introducción, en esta obra encontramos presentados temas que ya hemos visto representados en las obras que hemos analizado anteriormente. Varios ejemplos de ello son:

\begin{itemize}
	\item El \textbf{amor}: es el tema principal de la obra, como lo es también en \textit{El perro del hortelano} y en \textit{La dama boba}, ambas de Lope de Vega. Sin embargo, esta obra se asemeja más a \textit{La dama boba}, dado que el amor entre Romeo y Julieta se trata de un amor prohibido, porque provienen de familias rivales, con lo que se introduce el siguiente tema.
	\item El \textbf{honor}: puesto que los amantes provienen de familias rivales, ninguna de las dos familias aprobaría una relación de este tipo, ya que según la tradición de la época, perjudicaría su honor.
	\item El \textbf{odio} que sienten los Montesco hacia los Capuleto, y viceversa.
	\item El \textbf{arrepentimiento}: este aparece en menor medida, pero destaca sobre todo al final de la obra, cuando los padres se arrepienten al ver a sus hijos muertos.
\end{itemize}

\section{Relación con el contexto histórico}

La acción se desarrolla entre las ciudades italianas de Verona y Mantua, e intervienen en ella la familia de los Montesco y la de los Capuleto, aparentemente de naturaleza noble o burguesa, lo que, junto a la aparición de criados, nos recuerda una vez más el modelo de sociedad estamental. Muestra de dicha naturaleza noble de estas familias es la importancia que se le da al honor, propio de las clases privilegiadas.\\

Fuera de esto, como no hay muchos personajes, la acción se mantiene sencilla y sigue un único camino, lo que facilita la comprensión por parte del público. Sin embargo, esto dificulta las posibilidades de identificar más relaciones con el contexto histórico de la época, hacia finales del siglo XVI en un escenario italiano, por lo que el contexto no debería variar mucho con respecto al español de aquel entonces, dada la cercanía y la influencia de la corona de Aragón desde el sur de la península Itálica, hacia la zona de Nápoles.

\end{document}