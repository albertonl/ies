\documentclass[12pt,a4paper]{article}
\usepackage[spanish]{babel}
\usepackage[utf8]{inputenc}

\usepackage[right=2cm,left=3cm,top=2cm,bottom=2cm,headsep=0cm,footskip=0.5cm]{geometry}
\usepackage[dvips]{graphicx}
\usepackage{amsmath}
\usepackage{underscore}

\title{\textbf{La lírica del barroco: el conceptismo y el culteranismo}}
\author{Alberto Navalón Lillo}

\begin{document}

\maketitle
\tableofcontents

\section{El conceptismo en Quevedo}


\indent\textit{Piedra soy en sufrir pena y cuidado}\\
\indent\textit{y cera en el querer enternecido,}\\
\indent\textit{sabio en amar dolor tan bien nacido,}\\
\indent\textit{necio en ser en mi daño porfiado,}\\

\indent\textit{medroso en no vencerme acobardado}\\
\indent\textit{y valiente en no ser de mí vencido,}\\
\indent\textit{hombre en sentir mi mal, aun sin sentido,}\\
\indent\textit{bestia en no despertar desengañado.}\\

\indent\textit{En sustentarme entre los fuegos rojos,}\\
\indent\textit{en tus desdenes ásperos y fríos,}\\
\indent\textit{soy salamandra, y cumplo tus antojos;}\\

\indent\textit{y las niñas de aquestos\footnote{\textit{aquesto}: aqueste (neutro: \textit{esto}).} ojos míos}\\
\indent\textit{se han vuelto, con la ausencia de tus ojos,}\\
\indent\textit{ninfas que habitan dentro de dos ríos.}\\

Siguiendo su costumbre, Francisco de Quevedo escribe el poema 379 siguiendo el modelo conceptista, al que habitualmente recurre. Esta composición está estructurada como soneto, y desarrolla brevemente y de forma densa, en cuanto a la acumulación de ideas, la capacidad del yo poético de soportar el estar enamorado y no ser correspondido. Todo esto se explica metafóricamente. En términos generales, se podría decir que la totalidad del poema es una gran metáfora, dentro de la cual observamos otros recursos, como las hipérboles, ya que exagera mucho los sentimientos de amor y sufrimiento (\textit{En sustentarme entre los fuegos rojos,} / \textit{en tus desdenes ásperos y fríos,}), otras metáforas (por ejemplo: Piedra soy en sufrir pena y cuidado; aquí se introduce el término piedra para representar la gran resistencia ante las penas); y por último, una gran cantidad de antítesis.\\

A lo largo de los dos primeros cuartetos, se enumeran una serie de antítesis, de dos versos cada una, con las que Quevedo se describe. Un ejemplo de esto es:\\

\indent \textit{[...] sabio en amar dolor tan bien nacido}\\
\indent \textit{necio en ser en mi daño porfiado, [...]}\\

Aquí podemos observar cómo se califica el yo poético, que indirectamente corresponde al mismo Quevedo, de \textit{sabio} al \textit{amar dolor tan bien nacido}, esto es, al amar a la dama de la cual tan enamorado está, que, no obstante, tanto dolor le causa; y de \textit{necio} al ser \textit{en su daño porfiado}, es decir, al continuar conservando ese sentimiento [el amor a la dama], que tanto le duele, además, a sabiendas de ello. Como esta antítesis de dos versos, podemos encontrar un total de cuatro en el poema, que corresponden con los versos 1-2, 3-4 (primer cuarteto), 5-6, y 7-8 (segundo cuarteto). Con estas antítesis también representa Quevedo de una forma ingeniosa las características contradictorias del amor, que en este caso corresponde con el denominado concepto de este poema, común en la lírica conceptista.\\

En comparación con el modelo culteranista, en el modelo conceptista se hace un uso mucho más abundante de la retórica, dada la intención de los autores conceptistas como Quevedo de explicar el ya explicado concepto de una forma ingeniosa, para exhibir las capacidades del poeta, aunque más indirectamente, a diferencia de en el culteranismo, en que se hace de la forma más visible posible.\\

% \footnotetext{\textit{aquesto}: aqueste (neutro: \textit{esto}).} % aquestos (v. 12)

\subsection{Referencias}

\textbf{Ortega}, M. L. (1997). \textit{La poésie amoureuse de Quevedo}. ENS Éditions, OpenEdition Books.\\
https://books.openedition.org/enseditions/2231?lang=es\\

\textbf{Arellano}, I. (1997). ``La amada, el amante y los modelos amorosos en la poesía de Quevedo''. Instituto Cervantes, Universidad de Navarra.\\
http://www.cervantesvirtual.com/obra-visor/la-amada-el-amante-y-los-modelos-amorosos-en-la-poesa-de-quevedo-0/html/01772a92-82b2-11df-acc7-002185ce6064_3.html\\

\textbf{Pozuelo} Yvancos, J. ``Aproximación a la poesía amorosa de Quevedo''. Centro Virtual Cervantes, Instituto Cervantes.\\
https://cvc.cervantes.es/literatura/quevedo_critica/p_amorosa/pozuelo.htm\\

\textbf{Alonso} Veloso, M. J. (2008). ``La poesía de Quevedo no incluida en las ediciones de 1648 y 1670: una propuesta acerca de la ordenación y el contenido de la «Musa décima»'', p. 19. Universidad de Vigo.\\
https://minerva.usc.es/xmlui/bitstream/handle/10347/16219/2008PERINOLAALONSOVELOSOMUSADECIMAQUEVEDO.pdf?sequence=1\&isAllowed=y\\

\textbf{De Toledo} y Godoy, I. (1950). \textit{Cancionero antequerano}. Instituto Cervantes.\\

Varios autores (2005-2020). ``Conceptismo''. Wikipedia.\\
https://es.wikipedia.org/wiki/Conceptismo\\

Real Academia Española. (2014). \textit{Diccionario de la lengua española} (23.ª ed.).\\
https://dle.rae.es\\

\section{El culteranismo en Góngora}


\indent\textit{Purpúreas rosas sobre Galatea}\\
\indent\textit{la Alba entre lirios cándidos deshoja:}\\
\indent\textit{duda el amor cuál más su color sea,}\\
\indent\textit{o púrpura nevada, o nieve roja.}\\
\indent\textit{De su frente la perla es, eritrea\footnote{\textit{eritreo}: perteneciente o relativo al mar Rojo o a los territorios que baña. (indirectamente representa un color oscuro).},}\\
\indent\textit{émula\footnote{\textit{émulo}: competidor o imitador de alguien o de algo, procurando excederlo o aventajarlo.} vana. El ciego Dios se enoja,}\\
\indent\textit{y, condenando su esplendor, la deja}\\
\indent\textit{pender en oro al nácar de su oreja.}\\

Luis de Góngora, como creador del modelo lírico culteranista, hace uso de él en varias composiciones a lo largo de los cinco últimos años de su vida, entre las cuales se encuentra la composición aquí presentada. Con una estructura monostrófica, desarrolla la idea de la soberbia, aplicada a la perla, que intenta superar el esplendor de la ninfa Galatea, y de cómo los dioses castigan este tipo de pecados, en este caso, convirtiendo a la perla en pendiente. Todo esto se da dentro de un entorno de la mitología de la Grecia Clásica. Por tanto, se está describiendo el esplendor insuperable de la ninfa, mientras se intenta dar una lección moral, y a su vez religiosa (se menciona el castigo de los dioses ante los pecados); mediante un lenguaje muy bello y complejo, y dentro de un lugar mitológico idealizado, o \textit{locus amoenus}.\\

De entre las características generales de la lírica culteranista, en este texto, que es la estrofa XIV de la \textit{Fábula de Polifemo y Galatea} (1612); encontramos varias metáforas puras, principalmente la del último verso (pender en oro al nácar de su oreja.), que representa un pendiente en la oreja de Galatea; y la que abarca los cuatro primeros versos, que se podría explicar como que la piel de Galatea parece hecha de lirios blancos, pero el rubor\footnote{\textit{rubor}: enrojecimiento del rostro provocado por la vergüenza.} de sus mejillas hace pensar que el Alba ha deshojado sus rosas (rojas), sobre esos lirios (blancos). A partir de esto, también observamos la presencia de una ornamentación sensorial del ver, dada la gran abundancia de adjetivos (principalmente colores).\\

En los textos culteranistas, el principal objetivo es transmitir muy poco contenido, decorado con un lenguaje lo más bello y complejo posible, para exhibir las capacidades del autor. Luis de Góngora a menudo recurría a la latinización de la sintaxis, mediante el uso del hipérbaton (\textit{Purpúreas rosas sobre Galatea} / \textit{la Alba entre lirios cándidos deshoja:}), y de construcciones típicas del latín. Un ejemplo de esto último es el uso del subjuntivo (usual en latín), en lugar del indicativo (más común del castellano), en algunas oraciones interrogativas indirectas (\textit{duda el amor cuál más su color sea}).\\

Por último, también cabe mencionar la evidente alusión a referentes mitológicos, y la intertextualidad con autores clásicos, en este caso, con Ovidio, ya que la \textit{Fábula de Polifemo y Galatea} recrea la historia de Polifemo en sus \textit{Metamorfosis}.

% \footnotetext{\textit{eritreo}: perteneciente o relativo al mar Rojo o a los territorios que baña. (indirectamente representa un color oscuro)} % eritrea (v. 5)
% \footnotetext{\textit{émulo}: competidor o imitador de alguien o de algo, procurando excederlo o aventajarlo.} % émula (v. 6)
% \footnotetext{\textit{rubor}: enrojecimiento del rostro provocado por la vergüenza.} % rubor (par. 2)

\subsection{Referencias}

\textbf{De Góngora} y Argote, L. (1612). \textit{Fábula de Polifemo y Galatea}. Instituto Cervantes.\\
http://www.cervantesvirtual.com/obra-visor/fabula-de-polifemo-y-galatea--0/html/fedcc184-82b1-11df-acc7-002185ce6064_2.html\\

\textbf{Ivorra} Castillo, C. ``XIV''. Universitat de València.\\
https://www.uv.es/ivorra/Gongora/Polifemo/14.htm\\

\textbf{Castro} Rodríguez, M. L. ``Los cuatro elementos en la \textit{Fábula de Polifemo y Galatea}''. Universidad Nacional Autónoma de México.\\
https://www.academia.edu/10254927/Los_cuatro_elementos_en_la_Fabula_de_Polifemo_y_Galatea\\

Varios autores (2005-2020). ``Culteranismo''. Wikipedia.\\
https://es.wikipedia.org/wiki/Culteranismo\\

Real Academia Española. (2014). \textit{Diccionario de la lengua española} (23.ª ed.).\\
https://dle.rae.es\\

\end{document}