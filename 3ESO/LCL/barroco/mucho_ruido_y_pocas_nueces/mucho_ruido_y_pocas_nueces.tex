\documentclass[12pt,a4paper]{article}
\usepackage[spanish]{babel}
\usepackage[utf8]{inputenc}

\usepackage[right=2cm,left=3cm,top=2cm,bottom=2cm,headsep=0cm,footskip=0.5cm]{geometry}
\usepackage{graphicx,subfigure}
\usepackage{amsmath}
\usepackage{underscore}
\usepackage{array}
\usepackage{yhmath}
\usepackage{adjustbox}
\usepackage{multirow}

\title{\textbf{Análisis de \textit{Mucho ruido y pocas nueces}}}
\author{Alberto Navalón Lillo}

\begin{document}

\maketitle
\tableofcontents

\section{Introducción}

\textit{Mucho ruido y pocas nueces} (del original, en inglés: \textit{Much Ado About Nothing}), es una comedia del dramaturgo inglés William Shakespeare, que se dice escrita entre los años 1598 y 1599, aunque apareció por primera vez publicada en el año 1600. También se incluyó en el \textit{First Folio}, publicado en el año 1623. De esta obra también se han hecho adaptaciones cinematográficas, como la del año 1993, dirigida por Kenneth Branagh, o la de 2012, dirigida por Joss Whedon, ambas con el mismo título que la comedia original. Al igual que \textit{Romeo y Julieta}, la obra \textit{Mucho ruido y pocas nueces}, debido al tiempo en que fue escrita, pertenece al teatro isabelino, por lo que contiene muchas de las características comunes entre las obras de este grupo.\\

El argumento de esta comedia gira en torno a un tema ya repetido en la historia de la literatura: una dama es desdeñada por su novio tras haber sido falsamente acusada de infidelidad, pero al final se demuestra su inocencia y vuelve a reunirse con su prometido. Además, como se explicará más adelante, a esta trama principal se le añade una segunda acción paralela, la cual engloba a dos personajes: Beatriz y Benedicto, que están en un enredo amoroso, lleno de malentendidos y engaños, tanto al otro como a uno mismo, que atraen con fuerza al lector o espectador, quien presencia cómo ambos son llevados a admitir el amor que sienten el uno por el otro.\\

La acción, al igual que en otras obras del dramaturgo inglés, como \textit{Otelo} o \textit{Romeo y Julieta}, se desarrolla en Italia, aunque en este caso, en la ciudad portuaria de Mesina, al noreste de la isla de Sicilia. Recordemos que, en aquel entonces, el sur de la península itálica, incluyendo a Sicilia, formaba parte de la Corona de Aragón, por lo que el contexto histórico no se espera muy diferente al del territorio de la Corona de Aragón en la península ibérica en la misma época, dado el intenso contacto entre ambos territorios.

\section{Características de la obra}

De la misma forma que en \textit{Romeo y Julieta} no encontramos ninguna figura con el rol de bufón, aquí encontramos otros personajes que, ya sea por su simpleza, o por cualquier otro motivo, desarrollan pasajes de naturaleza cómica, en los que se combinan el humor y, en algunos casos, figuras literarias como la hipérbole, que esencialmente intensifican el efecto cómico:\\

\noindent\textsc{Guardia 2.º}\\
\indent\textit{Llamad al preboste mayor. Hemos pillado aquí a la más peligrosa confabulación de la impudicia que nunca se haya cometido en los estados de la nación.}\\
\noindent\textsc{Guardia 1.º}\\
\indent\textit{Y en ella está metida una mujer llamada la Moda. Yo la conozco; va con unos rizos.}\\
\noindent\textsc{Conrado}\\
\indent\textit{Señores, señores...}\\
\noindent\textsc{Guardia 2.º}\\
\indent\textit{Ya nos diréis quién es esa Moda; yo os lo aseguro.}\\
\begin{flushright}
	\textsc{Shakespeare}, William. \textit{Mucho ruido y pocas nueces}. Ed. B (p. 109).\\
\end{flushright}

Este fragmento representa un toque cómico dentro de un momento tan serio como es una detención (la detención de Borachio y Conrado). No obstante, más que estos pequeños fragmentos textuales, es, en cambio, la acción paralela del enredo amoroso entre Beatriz y Benedicto la que aporta comicidad a la obra, quizá incluso de una forma subliminal. Es la propia naturaleza de esta acción la que la aporta, o más bien, la contiene, dado que las situaciones cómicas son prácticamente inherentes a esta clase de escenarios.\\

\subsection{Las acciones paralelas}

Este es un aspecto que rompe en parte con la costumbre de mantener la trama de la obra lo más sencilla posible, a la vez que profunda, de forma que prácticamente cualquiera pudiera comprender el mensaje esencial que se pretendía transmitir. En este caso, a pesar de que la acción general se mantiene más o menos sencilla y comprensible, se añaden, además de forma muy brillante, las acciones complementarias, como son la de ideación y ejecución de los planes de Don Juan de acusar falsamente a Hero de infiel y así arruinar su boda con Claudio, con objetivo de poder quedársela para él; y la de Beatriz y Benedicto. Esta última, en algunas partes de la obra, alcanza tal importancia respecto a la denominada como acción principal (la relación entre Hero y Claudio), que se podría decir que llega a sustituir a esta.

\subsection{Los temas de la obra}

El \textbf{amor} es el tema principal de esta obra, ya sea el presente entre Hero y Claudio, o entre Beatriz y Benedicto. Como hemos observado en las otras tres obras analizadas previamente, a pesar de ser dos de ellas de Lope de Vega y otra de Shakespeare, en todas ellas el amor es el tema principal. Si además observamos, aunque sea únicamente de forma superficial, otras obras de este período, podemos llegar a la conclusión de que existió una tendencia respecto a la creación de obras de naturaleza amorosa en la época.\\

Los \textbf{celos}, que van ligados al amor en la mayoría de casos, también están presentes en \textit{Mucho ruido y pocas nueces}. En este caso vienen de la mano de Don Juan, dado que los siente hacia Claudio, quien fuera futuro marido de Hero.\\

También hallamos \textbf{odio} como un derivado de los celos, o también se podría describir como tal lo que Claudio siente hacia Hero cuando es engañado por Don Juan y la toma por infiel.\\

El \textbf{honor}, un tema muy recurrente en la literatura de este período, también está presente en la obra. Este tema se presenta en la primera escena del cuarto acto, en la capilla de la iglesia, cuando Hero es acusada de infiel por el mismo y engañado Claudio. Tras esto, su padre, Leonato, se avergüenza de su propia hija, por la posibilidad de que pudiese afectar a su honor, completamente ajeno a la falsedad de las acusaciones.\\

Como en \textit{Romeo y Julieta}, los personajes se \textbf{arrepienten} de sus acciones y pensamientos hacia Hero cuando conocen la verdad. Este es un tema más bien secundario, a pesar de que juega un rol esencial en el desarrollo completo de la obra, al igual que los cuatro anteriores.

\subsection{Estilo de escritura}

El manuscrito original de Shakespeare está escrito de forma predominante en prosa, aunque también se pueden hallar fragmentos en verso, con los cuales se alcanza una visión decorosa y se dota de vitalidad al público. No obstante, algunas ediciones en castellano han sido traducidas en su totalidad en prosa.

\section{Relación con el contexto histórico}

La acción, como ya se ha mencionado anteriormente, se desarrolla en la ciudad italiana de Mesina, por lo que podemos esperar un escenario muy similar al que se podría dar en la España peninsular entre los períodos del Renacimiento tardío y el Barroco temprano. La totalidad de la trama se desarrolla en espacios muy limitados, y hay muy poco contacto con el exterior. Se puede deducir que la sociedad se organiza de forma estamental, dadas sus costumbres, el contexto histórico externo de esa época y, en menor medida, la preocupación por el honor propia de las clases privilegiadas del momento.\\

Es difícil determinar otros vínculos o relaciones dada la naturaleza de la acción, y todo lo que a esta engloba. Se podrían deducir si, por ejemplo, nos basamos en el contexto histórico de la Corona de Aragón a finales del siglo XVI.

\end{document}