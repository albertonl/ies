\documentclass[12pt,a4paper]{article}
\usepackage[spanish]{babel}
\usepackage[utf8]{inputenc}

\usepackage[right=2cm,left=3cm,top=2cm,bottom=2cm,headsep=0cm,footskip=0.5cm]{geometry}
\usepackage{graphicx,subfigure}
\usepackage{amsmath}
\usepackage{underscore}
\usepackage{array}
\usepackage{yhmath}
\usepackage{adjustbox}
\usepackage{multirow}

\title{\textbf{Análisis de \textit{La dama boba}}}
\author{Alberto Navalón Lillo}

\begin{document}

\maketitle
\tableofcontents

\section{Introducción}

\textit{La dama boba} es una comedia escrita por Lope de Vega, firmada a 28 de abril de 1613, por lo cual se incluye en el periodo barroco, tanto por la época como por las características que presenta. Esta obra fue muy apreciada por el dramaturgo por los contenidos expresados, que, en diferentes niveles, contribuyen a configurar su entero universo poético y literario, además de ser escrita la obra en plena madurez creadora. Esta obra se ambienta en la España del siglo XVI, más concretamente, entre Illescas (Toledo) y Madrid, y presenta muchas de las características de la literatura y el teatro barroco, que fueron en gran parte concebidas por Lope, además de estar presentes de forma muy explícita, incluso llegando en parte a ser un poco antinatural. Esto, posiblemente se deba a que el desarrollo de Lope en la literatura como escritor, poeta y dramaturgo esté todavía en una fase temprana, y aún se ciña a las características ya establecidas, sin llegar a jugar con ellas y modificarlas a su antojo, con objetivo de crear un contexto y unos personajes más dinámicos, como observamos en parte en otras comedias como \textit{El perro del hortelano}. No obstante, ya encontramos a dos personajes dinámicos en la obra: Finea y Nise. Este tema se tratará con más profundidad tras la introducción.\\

El mensaje explícito de esta comedia es la capacidad educativa del amor, que consigue transformar a una dama boba en una dama discreta e inteligente. Esta es una idea neoplatónica, que originalmente se presenta como «la capacidad del amor para abrir el entendimiento». La totalidad de este mensaje gira en torno a una de las dos hermanas, Finea, que al principio era o se hacía la boba y que, una vez empieza a sufrir los efectos del amor, la razón y el entendimiento se apoderan de ella, todo ello en un sentido positivo.

\section{Análisis de la obra}

En este apartado se van a explicar los diferentes aspectos de la obra, como los personajes y sus roles, o el tema de la comedia, así como las características de la literatura y el teatro barroco presentes en ella.

\subsection{Personajes, roles y ejemplos de «personajes tipo»}

En la obra encontramos un total de 15 personajes, de los cuales:
\begin{itemize}
	\item Dos son protagonistas.
	\item Seis son secundarios.
	\item Los siete restantes son terciarios, dado que su implicación en el desarrollo de la trama de la obra es muy escaso y apenas apreciable.
\end{itemize}

Todo esto, en adición a un grupo de músicos (anónimos), que se pueden considerar como terciarios, o incluso eliminar cualquier tipo de clasificación, ya que no influyen en el desarrollo de la obra. A continuación podemos observar una tabla con todos los personajes:

\begin{table}[h]
	\centering
	\begin{tabular}{| m{5cm} | m{5cm} | m{5cm} |}
		\hline
		\textsc{Personaje}			 		& \textsc{Rol}								& \textsc{Categoría}			\\\hline
		\hline
		Nise 								& \textit{Dama}								& Protagonista 					\\\hline
		\textbf{Finea}						& \textit{Hermana de Nise}					& Protagonista 					\\\hline
		Otavio								& \textit{Viejo (padre)}	 				& Secundario 					\\\hline
		Liseo 								& \textit{Caballero}						& Secundario 					\\\hline
		Laurencio							& \textit{Caballero}						& Secundario 					\\\hline
		Pedro 								& \textit{Lacayo}							& Secundario 					\\\hline
		Clara 								& \textit{Criada}							& Secundario 					\\\hline
		Celia 								& \textit{Criada}							& Secundario 					\\\hline
		Turín 								& \textit{Lacayo}							& Terciario 					\\\hline
		Miseno 								& \textit{Amigo de Otavio} 					& Terciario 					\\\hline
		Leandro 							& \textit{Caballero} 						& Terciario 					\\\hline
		Duardo 								& \textit{Caballero} 						& Terciario 					\\\hline
		Feniso 								& \textit{Caballero}						& Terciario 					\\\hline
		Rufino								& \textit{Maestro} 							& Terciario 					\\\hline
		\textit{(Un Maestro de danzar)}		& 											&								\\\hline
		\textit{(Músicos)}					&											&								\\\hline
	\end{tabular}
	% \begin{tabular}{| >{\centering\arraybackslash}m{5cm} | >{\centering\arraybackslash}m{5cm} | >{\centering\arraybackslash}m{5cm} | }
	\caption{Clasificación según \textit{Fuenteovejuna; El caballero de Olmedo; La dama boba}\protect\footnotemark}
	\label{tab:1}
\end{table}

Entre estos personajes, podemos observar varios tipos de los denominados \textit{personajes tipo}, los cuales, esencialmente, no varían ni evolucionan a lo largo de la acción dramática.\\

En primer lugar, destacan las figuras del \textbf{galán} y la \textbf{dama}. Normalmente, estas dos figuras se representan muy relacionadas, y en esta obra encontramos dos galanes: Laurencio y Liseo; que se relacionan con las damas Finea y Nise, respectivamente; aunque en algunos momentos, estos dos vínculos se encuentran trocados.\\

A continuación, también hallamos las del \textbf{gracioso} y la \textbf{criada}. Estos son los criados del galán y de la dama, respectivamente: figuras presentadas en el párrafo anterior. También están vinculados amorosamente, y en la obra se corresponden con Pedro, lacayo y criado de Laurencio; y Clara, criada y confidente de Finea. Por la parte de Liseo y Nise no encontramos ningún vínculo gracioso-criada.\\

\footnotetext{\textbf{De Vega Carpio}, Lope. \textit{Fuenteovejuna; El caballero de Olmedo; La dama boba}, S.A. de Promoción y Ediciones. Madrid, 1999. (ISBN: 84-7461-197-0)}

En último lugar, encontramos la figura del \textbf{caballero}, Otavio, padre de las damas, que intenta restituir el honor de la familia, aunque los personajes implicados (Finea, Nise, Laurencio y Liseo en mayor medida) hagan caso omiso, lo que lleva a un final feliz, característica también típica de la obra dramática de Lope. En lo que respecta al resto de personajes, no encontramos ningún rol que corresponda con el de los «personajes tipo». Esto se debe a que, dada su escasa, casi nula influencia en el desarrollo de la acción principal, el enredo amoroso de las dos damas y sus pretendientes, no se les otorgan características muy elaboradas, por el simple hecho de mantener la acción lo más sencilla posible.

\subsection{Temas del teatro lopesco en la obra}

Al igual que en \textit{El perro del hortelano}, la acción principal está construida a partir de tres pilares, que son los temas fundamentales de la obra. Estos son el \textbf{amor}, los \textbf{celos} y el \textbf{honor}. El amor es el tema principal con bastante diferencia, ya que la acción se podría resumir en un enredo amoroso, lo cual también introduce el concepto de los celos, que también son un tema recurrente, tanto en esta como en otras muchas obras de Lope. En este caso, son Finea y Nise, que se pelean por Laurencio; y en parte Laurencio y Liseo, que en algún punto se llegan a pelear por Nise.\\

Finalmente, el honor está claramente presente, ya que encontramos entre los «personajes tipo» presentes en la obra la figura del caballero, que lucha (no necesariamente mediante conflictos físicos) por restituir el honor de la familia. En la obra, esta figura es Otavio, que pretende impedir que Finea se case con Laurencio, un hombre de estatus social inferior, a pesar de que realmente haya un amor latente y pasional entre ellos, cosa que devuelve a Finea el entendimiento.

\section{Relación con el contexto histórico}

Como se ha mencionado anteriormente, Otavio pretendía impedir que Finea se casara con Laurencio, dado que este era de condición social inferior. Con esto, recordamos la sociedad estamental, que aún estaba presente en aquella época. También podemos considerar a Laurencio como una especie de pícaro, figura típica y recurrente en la literatura del momento (recordemos que en el siglo XVI surgió la idea de la novela picaresca, con \textit{El Lazarillo de Tormes}), ya que es evidente que es únicamente el inmenso dote resultante de casarse con la dama boba, Finea, lo que le mueve a perseguir esa idea. A pesar de ello, a medida que nos acercamos al final de la obra, observamos un cambio en la actitud de Laurencio en lo que respecta a amar a Finea, que parece volverse real y no más fingida, por lo que se menciona la existencia de ese amor latente y pasional.\\

Esta actitud picaresca por parte de Laurencio es lo que nos lleva a pensar en el periodo de crisis social y política en la España del siglo XVI y XVII, que tiene un impacto notorio en las masas, como la miseria, las hambrunas o la pobreza extrema, por lo que hacen lo que fuera necesario para conseguir recursos, como, en este caso, persuadir a una dama con tal de conseguir un beneficio económico.\\

Fuera de esto, no podemos encontrar muchos más datos que referencien aspectos del contexto histórico del barroco, ya que la acción, así como los espacios y los personajes, son sencillos, con tal de que el espectador no tenga problemas a la hora de entender la obra. Por tanto, pocos más datos se pueden ver fuera de los ya mencionados.

\end{document}