\documentclass[12pt,a4paper]{article}
\usepackage[spanish]{babel}
\usepackage[utf8]{inputenc}

\usepackage[right=2cm,left=3cm,top=2cm,bottom=2cm,headsep=0cm,footskip=0.5cm]{geometry}
\usepackage[dvips]{graphicx}
\usepackage{amsmath}

\title{\textbf{Héctor de Troya: en la literatura y mitología}}
\author{Alberto Navalón Lillo}

\begin{document}

\maketitle
\tableofcontents

\section{Mitología}

\subsection{Contraataque troyano}
Los griegos toman el campamento de los troyanos, y los expulsan. Héctor decide tomar el campamento y quemar los barcos griegos al día siguiente\footnotemark. Entonces, Agamenón reúne a los griegos y batallan con los troyanos. Héctor solamente entra en batalla una vez que Agamenón se retira, herido.\\

Después, los troyanos invaden el campamento y la batalla se desata dentro. Héctor cae golpeado por Áyax con una piedra, pero Apolo baja del Olimpo y le infunde fuerzas. Finalmente, los troyanos alcanzan el barco de Protéstilas y ordena incendiarlo, pero Áyax lo impide\footnotemark.\\

Tras el furioso ataque de los troyanos, los griegos quieren que Aquiles vuelva a luchar, pero este se niega. Entonces, Patroclo toma la armadura de Aquiles y entra en combate con los troyanos, únicamente para ser matado posteriormente por Héctor\footnotemark. Aquiles, al enterarse de la muerte de Patroclo, promete venganza y vuelve a la lucha.

\footnotetext{\textit{Ilíada}, libro VIII} % Héctor decide tomar el campamento y quemar los barcos
\footnotetext{\textit{Ilíada}, libro XV} % Áyax impide la quema de los barcos por parte de los troyanos
\footnotetext{\textit{Ilíada}, libro XVI} % Héctor mata a Patroclo

\subsection{La última batalla de Héctor}
Los troyanos aconsejan volver a Troya para protegerse de la embestida griega. Sin embargo, Héctor ordena mantenerse en el campamento para esperar el enfrentamiento de Aquiles.\\

Al día siguiente, los griegos comienzan a presionar a los troyanos hacia la ciudad y, asustado, Héctor se esconde entre las tropas. Poco después, Aquiles mata a Polidoro, el hermano de Héctor. Ahora se muestra decidido y sale a enfrentarse a Aquiles. Héctor queda fuera de las puertas de Troya y es perseguido por Aquiles. Tras dar tres vueltas completas a la ciudad, Héctor decide plantarle cara\footnotemark. Finalmente, Aquiles mata a Héctor	clavándole la lanza en el cuello.\\

Aquiles maltrata el cadáver de Héctor durante los siguientes doce días, pero los dioses consiguen mantenerlo intacto. Cuando los troyanos recuperan su cuerpo sin vida, ya se estaban realizando unos funerales en su honor\footnotemark. Según lo contado en la \textit{Ilíada}, todo lo que le ocurrió a Héctor fue obra de los dioses.

\footnotetext{\textit{Ilíada}, libro XXII}
\footnotetext{\textit{Ilíada}, libro XXIV}

\section{Héctor en la literatura}
Gran parte de la historia de Héctor en la guerra de Troya está contada en la \textit{Ilíada}, que, compuesta por 15693 versos, narra los acontecimientos ocurridos durante 51 días en el décimo y último año de la guerra de Troya. Es una epopeya escrita por el poeta griego Homero, y se considera predecesora de las conocidas novelas de caballerías renacentistas.\\

Además, también podemos encontrar el nombre de Héctor de Troya en otras obras como la \textit{Divina comedia}\footnotemark, de Dante Alighieri; o en el \textit{Orlando furioso}\textit, de Ludovico Ariosto. En ambas obras, la figura de Héctor aparece como un recuerdo mitológico, siguiendo así la corriente renacentista de copiar los modelos grecolatinos, y eso, claro está, también incluye su mitología.

\footnotetext{En la \textit{Divina comedia}, Héctor y su familia se encuentran en el limbo. Véase: \textit{Infierno} (o \textit{Inferno}) en la \textit{Divina comedia}.}
\footnotetext{En el \textit{Orlando furioso}, se dice que la espada de Roldán perteneció antiguamente a Héctor de Troya.}

\end{document}