\documentclass[12pt,a4paper]{article}
\usepackage[spanish]{babel}
\usepackage[utf8]{inputenc}

\usepackage[right=2cm,left=3cm,top=2cm,bottom=2cm,headsep=0cm,footskip=0.5cm]{geometry}
\usepackage[dvips]{graphicx}
\usepackage{amsmath}
\usepackage{datetime}

\title{\textbf{El aparato respiratorio}}
\author{Alberto Navalón Lillo}
\newdate{date}{29}{01}{2020}
\date{\displaydate{date}}

\begin{document}

\maketitle
\tableofcontents

\section*{Introducción}
\addcontentsline{toc}{section}{Introducción}

El \textbf{aparato respiratorio} es el conjunto de órganos que regula el intercambio de gases en una gran variedad de seres vivos, por ejemplo, los humanos. Dentro de este aparato, podemos agrupar estos órganos en dos categorías: las \textbf{vías respiratorias}, y los \textbf{pulmones}.\\

Las \textbf{vías respiratorias} es el conjunto de conductos por los cuales viajan los gases para ser absorbidos dentro de los pulmones, siendo consecuentemente provenientes del exterior del organismo; o bien para ser expulsados desde los pulmones hasta el exterior. Todos estos conductos son de carácter tubular, ya que están optimizados para el paso de los gases durante su intercambio.\\

Los \textbf{pulmones} son los únicos dos órganos en el aparato respiratorio con una función diferente a permitir el paso de los gases. Podemos encontrar dos pulmones en el organismo humano, que funcionan a la par. Es en estos órganos donde se absorbe el oxígeno obtenido al inhalar aire del exterior, siendo este el único recurso que el organismo necesita de los múltiples que ofrece el aire (nitrógeno, dióxido de carbono, argón...). Desde los pulmones, el oxígeno absorbido pasa a la sangre, para ser distribuido a todas las células del cuerpo. De igual manera, las sustancias gaseosas perjudiciales presentes en la sangre, como el dióxido de carbono, fruto de la respiración celular, pasa a los pulmones para ser expulsado del organismo.

\section{Preparación}

\subsection{Materiales}

El material necesario para la realización de esta práctica es el siguiente:
\begin{itemize}
	\item Un aparato respiratorio de animal (preferiblemente de cordero o cerdo, debido a su manejable tamaño). A ser posible, se debería preservar la mayor parte de las vías respiratorias, para mejor visualización del aparato.
	\item Tijeras de disección, así como pinzas o incluso un bisturí.
	\item Una bandeja limpia para depositar el aparato respiratorio (por razones de higiene).
	\item Guantes desechables de látex o nitrilo (por razones de higiene).
	\item Varias cañitas de plástico.
\end{itemize}

\end{document}